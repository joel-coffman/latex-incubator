% \iffalse meta-comment
%
% Copyright (C) 2021 by Joel Coffman
%
% This file may be distributed and/or modified under the
% conditions of the LaTeX Project Public License, either version 1.2
% of this license or (at your option) any later version.
% The latest version of this license is in:
%
%   http://www.latex-project.org/lppl.txt
%
% and version 1.2 or later is part of all distributions of LaTeX
% version 1999/12/01 or later.
%
% \fi
%
% \iffalse
%<*driver>
\documentclass{ltxdoc}

\usepackage{enumitem}
\usepackage{hyperref}
\usepackage{microtype}
\usepackage{minted}

\usepackage{email}
\usepackage{minted-doc}

\usepackage{visualization}


% enumitem
\setlist{
  noitemsep,
}

% minted
\setminted{
  autogobble,
  breaklines,
}

% macros / environments
\newcommand{\palette}[2]{%
  \begin{tikzpicture}
    \begin{axis}[
        height=1in,
        hide axis,
        stack plots=x,
        #1,
    ]
      \foreach \i in {1,2,...,#2} {
        \addplot+[line width=1ex] coordinates {(0,1) (1,1)};
      }
    \end{axis}
  \end{tikzpicture}%
}


% package version information
\input{.version}


\EnableCrossrefs
\CodelineIndex
\RecordChanges
\begin{document}
  \DocInput{visualization.dtx}
\end{document}
%</driver>
% \fi
%
% \CheckSum{0}
%
% \CharacterTable
% {Upper-case   \A\B\C\D\E\F\G\H\I\J\K\L\M\N\O\P\Q\R\S\T\U\V\W\X\Y\Z
% Lower-case    \a\b\c\d\e\f\g\h\i\j\k\l\m\n\o\p\q\r\s\t\u\v\w\x\y\z
% Digits        \0\1\2\3\4\5\6\7\8\9
% Exclamation   \!     Double quote  \"     Hash (number) \#
% Dollar        \$     Percent       \%     Ampersand     \&
% Acute accent  \'     Left paren    \(     Right paren   \)
% Asterisk      \*     Plus          \+     Comma         \,
% Minus         \-     Point         \.     Solidus       \/
% Colon         \:     Semicolon     \;     Less than     \<
% Equals        \=     Greater than  \>     Question mark \?
% Commercial at \@     Left bracket  \[     Backslash     \\
% Right bracket \]     Circumflex    \^     Underscore    \_
% Grave accent  \`     Left brace    \{     Vertical bar  \|
% Right brace   \}     Tilde         \~}
%
%
% \changes{0.1.0}{2021/07/10}{Initial version}
%
% \GetFileInfo{visualization.sty}
%
% \DoNotIndex{\#,\$,\%,\&,\@,\\,\{,\},\^,\_,\~,\ }
% \DoNotIndex{\@ne}
% \DoNotIndex{\advance,\begingroup,\catcode,\closein}
% \DoNotIndex{\closeout,\day,\def,\edef,\else,\empty,\endgroup}
% \DoNotIndex{\global,\let,\relax}
%
% \title{The \textsf{visualization} package\thanks{This
% document corresponds to \textsf{visualization~\fileversion-\version, dated \filedate.}}}
% \author{Joel Coffman\\\email{joel.coffman@usafa.edu}}
%
% \maketitle
%
% \begin{abstract}
% This package comprises data visualization libraries for use with \textsc{pgfplots}.
% Each library defines styles to ensure that graphics are attractive, informative, and---most importantly---consistent.
% \end{abstract}
%
% \section{Usage}\label{section:usage}
% This package defines libraries for specific types of visualizations; each library in turn defines various styles (using \mintinline{tex}{\pgfplotsset}).
% All the libraries defined by this package are loaded automatically with this package.
% Thus, a simple
% \begin{VerbatimOut}[gobble=1]{examples/usepackage.tex}
% \usepackage{visualization}
% \end{VerbatimOut}
% \inputminted{latex}{examples/usepackage.tex}
% is sufficient to use all the styles.
% Of course, each library may also be loaded individually using \mintinline{tex}{\usepgfplotslibrary}---e.g.,
% \begin{VerbatimOut}[gobble=1]{examples/usepgfplotslibrary.tex}
% \usepgfplotslibrary{categorical}
% \end{VerbatimOut}
% \inputminted{latex}{examples/usepgfplotslibrary.tex}
% loads the library for categorical plots (e.g., bar graphs).
%
% \section{Libraries}\label{section:libraries}
% This section describes the various libraries for data visualization with a variety of examples.
%
% \subsection{Colors}\label{section:colors}
% The \textsf{colors} library defines styles for sequential, diverging, and qualitative data.
% By default, the palette is both color blind and print friendly.
%
% \begin{description}
%   \item[sequential]
%   Use the sequential color palette for numeric or ordered data.
%   By default, the sequential color palette supports five distinct values:
%   \palette{visualization/colors/sequential}{5}
%
%   \item[diverging]
%   Use the diverging color palette for numeric or ordered data with a meaningful central value (e.g., 0).
%   The diverging color palette requires specifying the number of distinct values (e.g., five):
%   \palette{visualization/colors/diverging=5}{5}
%
%   \item[qualitative]
%   Use the qualitative palette for categorical data.
%   By default, the qualitative color palette supports four distinct values:
%   \palette{visualization/colors/qualitative}{4}
% \end{description}
%
% \subsection{Categorical}\label{section:categorical}
% The \textsf{categorical} library defines styles for categorical data.
%
% \subsubsection{Bar Graphs}
%
% \begin{VerbatimOut}[gobble=1]{examples/bar-graph-x.tex}
% \begin{tikzpicture}
%   \begin{axis}[
%       xbar,
%       bar graph/direction=horizontal,
%       enlarge y limits=0.5,
%       height=2in,
%       width=0.5\linewidth,
%       symbolic y coords={Green,Yellow},
%       xlabel=Height (inches),
%       xmin=0.0,
%       xmax=26.0,
%       y dir=reverse,
%       ylabel=Seed Color,
%       ytick=data,
%       yticklabel style={
%         rotate=90,
%       },
%       visualization/colors/qualitative=2,
%   ]
%     \addplot+ coordinates {
%       (12.429,Green) +- (0.385,0)
%       (13.034,Yellow) +- (0.669,0)
%     };
%     \addplot+ coordinates {
%       (16.903,Green) +- (0.521,0)
%       (19.277,Yellow) +- (0.859,0)
%     };
%
%     \legend{Treatment\textsubscript{1},Treatment\textsubscript{2}}
%   \end{axis}
% \end{tikzpicture}
% \end{VerbatimOut}
% \begin{VerbatimOut}[gobble=1]{examples/bar-graph-y.tex}
% \begin{tikzpicture}
%   \begin{axis}[
%       ybar,
%       bar graph/direction=vertical,
%       enlarge x limits=0.5,
%       height=2in,
%       width=0.5\linewidth,
%       symbolic x coords={Green,Yellow},
%       xlabel=Seed Color,
%       xtick=data,
%       ylabel=Height (inches),
%       ymin=0.0,
%       ymax=26.0,
%       visualization/colors/qualitative=2,
%   ]
%     \addplot+ coordinates {
%       (Green,12.429) +- (0,0.385)
%       (Yellow,13.034) +- (0,0.669)
%     };
%     \addplot+ coordinates {
%       (Green,16.903) +- (0,0.521)
%       (Yellow,19.277) +- (0,0.859)
%     };
%
%     \legend{Treatment\textsubscript{1},Treatment\textsubscript{2}}
%   \end{axis}
% \end{tikzpicture}
% \end{VerbatimOut}
% \inputminted[gobble=1]{latex}{examples/bar-graph-y.tex}
% produces\\
% \input{examples/bar-graph-x.tex}%
% \input{examples/bar-graph-y.tex}\\
%
% \subsubsection{Likert}
% Responses to Likert items may be visualized using a diverging stacked bar chart.  ^^A TODO: Add citation
% For example,
% \begin{VerbatimOut}[gobble=1]{examples/single-likert-item.tex}
% \begin{tikzpicture}
%   \begin{axis}[
%       likert=5,
%       enlarge y limits=false,  ^^A suppress axis enlargement
%       hide x axis,
%       width=\linewidth - 35pt,
%       y=\baselineskip,
%       ytick=data,
%       yticklabels={,,},
%   ]
%     \addplot[offset] coordinates {(0.500,0)};
%
%     \addplot+ coordinates {(0.040,0)  [1]};  % Strongly disagree
%     \addplot+ coordinates {(0.240,0)  [6]};  % Disagree
%     \addplot+ coordinates {(0.440,0) [11]};  % Neutral
%     \addplot+ coordinates {(0.240,0)  [6]};  % Agree
%     \addplot+ coordinates {(0.040,0)  [1]};  % Strongly agree
%   \end{axis}
% \end{tikzpicture}
% \end{VerbatimOut}
% \inputminted[gobble=1]{latex}{examples/single-likert-item.tex}
% produces\\
% \input{examples/single-likert-item.tex}\\
% Of course, a ``bare'' graph is unlikely to be of much use without the additional context provided by a legend and scale.
% For example,
% \begin{VerbatimOut}[gobble=1]{examples/single-likert-item-with-legend-and-scale.tex}
% \begin{tikzpicture}
%   \begin{axis}[
%       likert=5,
%       enlarge y limits=false,  ^^A suppress axis enlargement
%       width=\linewidth - 35pt,
%       y=\baselineskip,
%       ytick=data,
%       yticklabels={,,},
%   ]
%     \addplot[offset] coordinates {(0.500,0)};
%
%     \addplot+ coordinates {(0.040,0)  [1]};  % Strongly disagree
%     \addplot+ coordinates {(0.240,0)  [6]};  % Disagree
%     \addplot+ coordinates {(0.440,0) [11]};  % Neutral
%     \addplot+ coordinates {(0.240,0)  [6]};  % Agree
%     \addplot+ coordinates {(0.040,0)  [1]};  % Strongly agree
%
%     \legend{Strongly disagree,Disagree,Neutral,Agree,Strongly agree};
%   \end{axis}
% \end{tikzpicture}
% \end{VerbatimOut}
% \inputminted[gobble=1]{latex}{examples/single-likert-item-with-legend-and-scale.tex}
% produces\\
% \input{examples/single-likert-item-with-legend-and-scale.tex}\\
% Responses may be compared (e.g., pre- and post-treatment) by stacking them vertically.
% For example,
% \begin{VerbatimOut}[gobble=1]{examples/single-likert-item-comparison.tex}
% \begin{tikzpicture}
%   \begin{axis}[
%       likert=5,
%       explicit axis labels={Disagree}{Agree},
%       symbolic y coords={Post,Pre},
%       width=\linewidth - 35pt,
%       ytick=data,
%   ]
%     \addplot[offset] coordinates {
%       (0.250,Pre)
%       (0.750,Post)
%     };
%
%     \addplot+ coordinates {
%       (0.125,Pre) [1]
%       (0.000,Post) [0]
%     };  % Strongly disagree
%     \addplot+ coordinates {
%       (0.375,Pre) [3]
%       (0.125,Post) [1]
%     };  % Disagree
%     \addplot+ coordinates {
%       (0.500,Pre) [4]
%       (0.250,Post) [2]
%     };  % Neutral
%     \addplot+ coordinates {
%       (0.000,Pre) [0]
%       (0.500,Post) [4]
%     };  % Agree
%     \addplot+ coordinates {
%       (0.000,Pre) [0]
%       (0.125,Post) [1]
%     };  % Strongly agree
%   \end{axis}
% \end{tikzpicture}
% \end{VerbatimOut}
% \inputminted[gobble=1]{latex}{examples/single-likert-item-comparison.tex}
% produces\\
% \input{examples/single-likert-item-comparison.tex}
%
% \StopEventually{
%   \PrintChanges
%   \PrintIndex
% }
%
% \appendix
%
% \iffalse
%<*package>
% \fi
%
% \section{Implementation}
% This section documents the implementation of the package.
%
% Require \LaTeXe.
%    \begin{macrocode}
\NeedsTeXFormat{LaTeX2e}
%    \end{macrocode}
% Identify package and version.
%    \begin{macrocode}
\ProvidesPackage{visualization}[%
    2023/07/17 %
    v0.1.2 %
    Data visualization library%
]
%    \end{macrocode}
%
% \subsection{Packages}
% Load packages required by this one.
%    \begin{macrocode}
\RequirePackage{pgfplots}
%    \end{macrocode}
%
% \subsubsection{Libraries}
% Load the libraries defined by this package.
%    \begin{macrocode}
\usepgfplotslibrary{categorical}
\usepgfplotslibrary{colors}
\usepgfplotslibrary{generic}
%    \end{macrocode}
% \changes{0.1.1}{2022/07/28}{
%    Add generic library
% }
%
% \subsection{Macros}
% This section describes the macros in the \textsf{visualization} package.
%
% \iffalse
%</package>
% \fi
%
% \subsection{Libraries}
%
% \iffalse
%<*tikzlibrarypgfplots.categorical.code.tex>
% \fi
%
% \subsubsection{Categorical}
% This section documents the implementation of the \textsf{categorical} library.
%
%    \begin{macrocode}
\RequirePackage{relsize}

\usepgfplotslibrary{colors}

\pgfplotsset{
  compat=1.16,
  % styles
  bar graph/.style={
    axis x line=bottom,
    axis y line*=left,
    error bars/error bar style={
      black,
      opacity=0.5,
    },
    every axis plot/.append style={
      draw,
      fill,
      fill opacity=0.9,
    },
    legend cell align=left,
    legend style={
      anchor=south,
      at={(0.5,1.0)},
      draw=none,  % no bounding box
      fill=none,  % transparent background
      font=\smaller,
      legend columns=-1,
      /tikz/every even column/.append style={
        column sep=0.25em,
      },
    },
    nodes near coords={
      \pgfmathprintnumber[
          fixed zerofill,
          precision=3,
      ]\pgfplotspointmeta%
    },
    nodes near coords style={
      color=black,
      font=\smaller[2],
      text opacity=0.85,
    },
    tick label style={
      font=\smaller,
    },
    typeset ticklabels with strut,
    x axis line style=-,
    xlabel near ticks,
    ylabel near ticks,
    visualization/colors/qualitative,
  },
  bar graph/direction/.is choice,
  bar graph/direction/horizontal/.style={
    bar graph,
    error bars/x dir=both,
    error bars/x explicit,
    nodes near coords style={
      yshift=0.75ex,
    },
  },
  bar graph/direction/vertical/.style={
    bar graph,
    error bars/y dir=both,
    error bars/y explicit,
    nodes near coords align=horizontal,
    nodes near coords style={
      rotate=90,
      yshift=0.75ex,  % must appear *after* rotate
    },
  },
  diverging stacked/.style={
    stack negative=separate,
    bar graph,
    enlarge y limits=0.75,
    every axis plot/.append style={
      draw,
      fill,
      fill opacity=0.9,
      typeset ticklabels with strut,
    },
    nodes near coords,
    nodes near coords style={
    },
    offset/.style={
      draw=none,
      fill=none,
      forget plot,
    },
    visualization/colors/diverging=#1,
  },
  likert/.style={
    xbar stacked,
    diverging stacked=#1,
    axis line style={
      opacity=0.50,
    },
    bar width=\baselineskip,
    enlarge y limits={abs=\baselineskip},
    explicit axis labels/.style 2 args={
      axis line style={stealth-stealth},
      after end axis/.append code={
        \begin{scope}[
          every node/.style={
            execute at end node=\strut,
            fill=white,
            fill opacity=0.9,
            font=\smaller,
            inner sep=1pt,
            text opacity=0.5,
          },
        ]
          \node[
              anchor=west,
              xshift=0.75em,
          ] at (axis description cs:0,0) {##1};
          \node[
              anchor=east,
              xshift=-0.75em,
          ] at (axis description cs:1,0) {##2};
        \end{scope}
      },
      xtick={1.00},
      xticklabels={},
    },
    explicit axis labels/.default={Disagree}{Agree},
    extra x ticks=1.00,
    extra x tick labels={},
    extra x tick style={
      grid=major,
    },
    label style={
      font=\smaller,
    },
    legend style={
      font=\smaller,
      legend columns=-1,
    },
    minor tick num=1,
    point meta=explicit symbolic,
    scale only axis,
    tick label style={
      font=\smaller,
    },
    xmin=0.00,
    xmax=2.00,
    xtick={0.00,0.50,1.00,1.50,2.00},
    xticklabels={$100\%$,$50\%$,$0\%$,$50\%$,$100\%$},
    x tick label style={
      text opacity=0.5,
    },
    y=1.3\baselineskip,
    y axis line style={
      draw=none,
    },
    y tick label style={
      anchor=west,
      xshift=-1.75em,
    },
    y tick style={
      draw=none,
    },
  },
}
%    \end{macrocode}
% \changes{0.1.2}{2023/07/13}{
%    Add minor tick marks to Likert visualization
% }
% \changes{0.1.2}{2023/07/13}{
%    Denote midpoint of Likert scale
% }
% \changes{0.1.2}{2023/07/13}{
%    Label x axis for Likert visualization
% }
%
% \iffalse
%</tikzlibrarypgfplots.categorical.code.tex>
% \fi
%
% \iffalse
%<*tikzlibrarypgfplots.colors.code.tex>
% \fi
%
% \subsubsection{Colors}
% This section documents the implementation of the \textsf{bargraph} library.
%
% Use colorblind and print friendly palettes from ColorBrewer\footnote{\url{https://colorbrewer2.org/}} when possible.
%    \begin{macrocode}
\usepgfplotslibrary{colorbrewer}

\pgfplotsset{
  visualization/colors/.is family,
  visualization/colors,
  diverging/.value required,
  diverging/.style={
    cycle list/RdYlBu-#1,
  },
  qualitative/.default=4,
  qualitative/.style={
    cycle list/Paired-#1,
  },
  sequential/.default=5,
  sequential/.style={
    cycle list/YlGnBu-#1,
  },
}
%    \end{macrocode}
%
% Define custom cycle lists for the qualitative palette.
% \textsc{pgfplots} does not define a ``Paired'' palette with only two colors.
% These palettes rearrange the colors to use light blue, light green, and light red, which appear to be colorblind safe.\footnote{\url{https://davidmathlogic.com/colorblind/}}
%
%    \begin{macrocode}
\pgfplotscreateplotcyclelist{Paired-2}{
  {Paired-A},
  {Paired-C}
}
\pgfplotscreateplotcyclelist{Paired-3}{
  {Paired-A},
  {Paired-C},
  {Paired-E}
}
\pgfplotsset{
  cycle list/Paired-2/.style={
    cycle list name=Paired-2,
  },
  cycle list/Paired-3/.style={
    cycle list name=Paired-3,
  },
}
%    \end{macrocode}
%
% \iffalse
%</tikzlibrarypgfplots.colors.code.tex>
% \fi
%
% \iffalse
%<*tikzlibrarypgfplots.generic.code.tex>
% \fi
%
% \subsubsection{Generic}
% This section documents the implementation of the \textsf{generic} library, which contains styles used for a variety of visualizations.
%
% Require packages used by this library.
%    \begin{macrocode}
\RequirePackage{relsize}
%    \end{macrocode}
%
% Define styles.
%    \begin{macrocode}
\pgfplotsset{
%    \end{macrocode}
%
% A subtitle appears beneath the axis title but in a smaller font.
%    \begin{macrocode}
  subtitle/.style={
    align=center,
    title/.add={}{\\{\smaller #1\par}},
  },
%    \end{macrocode}
%    \begin{macrocode}
}
%    \end{macrocode}
%
% \iffalse
%</tikzlibrarypgfplots.generic.code.tex>
% \fi
%
% \Finale
\endinput
