% \iffalse meta-comment
%
% Copyright (C) 2014 by Joel Coffman
% -----------------------------------
%
% This file may be distributed and/or modified under the
% conditions of the LaTeX Project Public License, either version 1.2
% of this license or (at your option) any later version.
% The latest version of this license is in:
%
%   http://www.latex-project.org/lppl.txt
%
% and version 1.2 or later is part of all distributions of LaTeX
% version 1999/12/01 or later.
%
% \fi
%
% \iffalse
%<package>\NeedsTeXFormat{LaTeX2e}
%<package>\ProvidesPackage{beamercolorthemewahoo}
%<package>  [2016/10/02 v0.1.0 Beamer color theme for the University of Virginia]
%
%<*driver>
\documentclass{ltxdoc}
\usepackage{beamerarticle}

\input{.version}
\usepackage{beamercolorthemewahoo}

\usepackage{minted}
\usepackage{url}

\newcommand{\testcolor}[1]{%
  \fboxsep0pt%
  \fbox{%
    \colorbox{#1}%
    {\phantom{XX}}%
  }%
}

\EnableCrossrefs
\CodelineIndex
\RecordChanges
\begin{document}
  \DocInput{beamercolorthemewahoo.dtx}
\end{document}
%</driver>
% \fi
%
% \CheckSum{0}
%
% \CharacterTable
% {Upper-case   \A\B\C\D\E\F\G\H\I\J\K\L\M\N\O\P\Q\R\S\T\U\V\W\X\Y\Z
% Lower-case    \a\b\c\d\e\f\g\h\i\j\k\l\m\n\o\p\q\r\s\t\u\v\w\x\y\z
% Digits        \0\1\2\3\4\5\6\7\8\9
% Exclamation   \!     Double quote  \"     Hash (number) \#
% Dollar        \$     Percent       \%     Ampersand     \&
% Acute accent  \'     Left paren    \(     Right paren   \)
% Asterisk      \*     Plus          \+     Comma         \,
% Minus         \-     Point         \.     Solidus       \/
% Colon         \:     Semicolon     \;     Less than     \<
% Equals        \=     Greater than  \>     Question mark \?
% Commercial at \@     Left bracket  \[     Backslash     \\
% Right bracket \]     Circumflex    \^     Underscore    \_
% Grave accent  \`     Left brace    \{     Vertical bar  \|
% Right brace   \}     Tilde         \~}
%
%
% \changes{0.1.0}{2016/10/02}{%
%   Initial version
% }
%
% \GetFileInfo{beamercolorthemewahoo.sty}
%
% \DoNotIndex{\#,\$,\%,\&,\@,\\,\{,\},\^,\_,\~,\ }
% \DoNotIndex{\@ne}
% \DoNotIndex{\advance,\begingroup,\catcode,\closein}
% \DoNotIndex{\closeout,\day,\def,\edef,\else,\empty,\endgroup}
% \DoNotIndex{\global,\let,\relax}
%
% \DoNotIndex{\if@beamercolorthemewahoo@gray,
%             \@beamercolorthemewahoo@graytrue}
%
% \title{
%   The \textsf{beamercolorthemewahoo} package\thanks{%
%     This document corresponds to \protect\textsf{beamerouterthemewahoo}~\fileversion-\version, dated \filedate.
%   }
% }
% \author{Joel Coffman\\\texttt{joel.coffman@jhu.edu}}
%
% \maketitle
%
% \begin{abstract}
% A Beamer color theme for the University of Virginia.
% The ``wahoo'' color theme defines and uses colors from the color palette of the University of Virginia.
% \end{abstract}
%
% \section{Usage}
% Per Beamer's documentation, this color theme should be loaded via the following command:
% \begin{minted}[
%   gobble=2,
% ]{latex}
  \usecolortheme{wahoo}
% \end{minted}
%
% \StopEventually{
%   \PrintChanges
%   \PrintIndex
% }
%
% \appendix
%
% \iffalse
%<*package>
% \fi
%
% \section{Implementation}
% See Beamer's documentation and the implementation of various themes for more information about the color palettes and color commands.
% The \mintinline{latex}{default} color theme\footnote{%
%   \url{http://mirrors.ctan.org/macros/latex/contrib/beamer/base/themes/color/beamercolorthemedefault.sty}
% } is a particularly good resource for the available color commands.
%
% \subsection{Colors}
% Define colors from the University of Virginia color palette.
%    \begin{macrocode}
\definecolor{Pantone 158}{RGB}{229,114,0}
\definecolor{Pantone 294}{RGB}{0,47,108}
\definecolor{Pantone 383}{RGB}{168,173,0}
%    \end{macrocode}
%
% \subsection{Theme}
% Override colors used by Beamer's color palettes.
%
% \mintinline{latex}{normal text} defines the most basic style for a presentation when another color palette does not apply.
%    \begin{macrocode}
\setbeamercolor{normal text}{
%    \end{macrocode}
% Use a white background.
%    \begin{macrocode}
  bg=white,
%    \end{macrocode}
% Use black foreground.
%    \begin{macrocode}
  fg=black,
}
%    \end{macrocode}
%
% \mintinline{latex}{alerted text} is an orange. \testcolor{Pantone 158}
%    \begin{macrocode}
\setbeamercolor{alerted text}{
  fg=Pantone 158,
}
%    \end{macrocode}
%
% \mintinline{latex}{example text} is a dark green. \testcolor{Pantone 383}
%    \begin{macrocode}
\setbeamercolor{example text}{
  fg=Pantone 383,
}
%    \end{macrocode}
%
% The \mintinline{latex}{structure} color palette is derived from a medium blue. \testcolor{Pantone 294}
% Shades of this color are used for most elements in the presentation.
%    \begin{macrocode}
\setbeamercolor{structure}{
  fg=Pantone 294,
}
%    \end{macrocode}
% \iffalse
%</package>
% \fi
%
% \Finale
