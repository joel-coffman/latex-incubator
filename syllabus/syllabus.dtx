% \iffalse meta-comment
%
% Copyright (C) 2016 by The Johns Hopkins University Applied Physics Laboratory
% -----------------------------------------------------------------------------
%
% This file may be distributed and/or modified under the
% conditions of the LaTeX Project Public License, either version 1.2
% of this license or (at your option) any later version.
% The latest version of this license is in:
%
%   http://www.latex-project.org/lppl.txt
%
% and version 1.2 or later is part of all distributions of LaTeX
% version 1999/12/01 or later.
%
% \fi
%
% \iffalse
%<*driver>
\documentclass{ltxdoc}

\usepackage{hyperref}
\usepackage{microtype}
\usepackage{minted}

\usepackage{email}
\usepackage{minted-doc}
\usepackage{syllabus}


% geometry
\geometry{pass}  % disable all geometry options for package documentation

% minted
\setminted{
  autogobble,
  breaklines,
}


% package version information
\input{.version}


\EnableCrossrefs
\CodelineIndex
\RecordChanges

\begin{document}
  \DocInput{syllabus.dtx}
\end{document}
%</driver>
% \fi
%
% \CheckSum{0}
%
% \CharacterTable
% {Upper-case   \A\B\C\D\E\F\G\H\I\J\K\L\M\N\O\P\Q\R\S\T\U\V\W\X\Y\Z
% Lower-case    \a\b\c\d\e\f\g\h\i\j\k\l\m\n\o\p\q\r\s\t\u\v\w\x\y\z
% Digits        \0\1\2\3\4\5\6\7\8\9
% Exclamation   \!     Double quote  \"     Hash (number) \#
% Dollar        \$     Percent       \%     Ampersand     \&
% Acute accent  \'     Left paren    \(     Right paren   \)
% Asterisk      \*     Plus          \+     Comma         \,
% Minus         \-     Point         \.     Solidus       \/
% Colon         \:     Semicolon     \;     Less than     \<
% Equals        \=     Greater than  \>     Question mark \?
% Commercial at \@     Left bracket  \[     Backslash     \\
% Right bracket \]     Circumflex    \^     Underscore    \_
% Grave accent  \`     Left brace    \{     Vertical bar  \|
% Right brace   \}     Tilde         \~}
%
%
% \changes{0.1.0}{2016/04/22}{Initial version}
%
% \GetFileInfo{syllabus.sty}
%
% \DoNotIndex{\#,\$,\%,\&,\@,\\,\{,\},\^,\_,\~,\ }
% \DoNotIndex{\@ne}
% \DoNotIndex{\advance,\begingroup,\catcode,\closein}
% \DoNotIndex{\closeout,\day,\def,\edef,\else,\empty,\endgroup}
% \DoNotIndex{\global,\let,\relax}
%
% \title{The \textsf{syllabus} package\thanks{This document
% corresponds to \textsf{syllabus}~\fileversion-\version,
% dated \filedate.}}
% \author{Joel Coffman\\\email{joel.coffman@jhu.edu}}
%
% \maketitle
%
% \begin{abstract}
% The \textsf{syllabus} package facilitates typesetting syllabi.
% This package is derived from the general syllabus template provided for faculty and staff in Johns Hopkins's Whiting School of Engineering.\footnote{%
%   \url{http://engineering.jhu.edu/about/faculty-staff-resources/}%
% }
% \end{abstract}
%
% \section{Usage}\label{section:usage}
% Loading this package in the document's preamble will override the style defined by the document class---i.e.,
% \begin{VerbatimOut}[gobble=2]{minted/usage.out}
% \usepackage{syllabus}
% \end{VerbatimOut}
% \inputminted{latex}{minted/usage.out}
% is sufficient.
%
% Use the |logo| macro to include the university logo in title.
% The logo must be defined prior to invoking \mintinline{latex}{\maketitle}.
%
% {
%   \bibliographystyle{apalike}
%   \bibliography{../references}
% }
%
% \StopEventually{
%   \PrintChanges
%   \PrintIndex
% }
%
% \appendix
%
% \iffalse
%<*package>
% \fi
%
% \section{Implementation}
% This section documents the implementation of the package.
%
% Require \LaTeXe.
%    \begin{macrocode}
\NeedsTeXFormat{LaTeX2e}
%    \end{macrocode}
% Identify package and version.
%    \begin{macrocode}
\ProvidesPackage{syllabus}[%
    2018/07/05 %
    v0.1.0 %
    Style file for syllabi%
]
%    \end{macrocode}
%
% \subsection{Packages}
% Declare package options.
%
% \paragraph{unnumbered}
% The \mintinline{latex}|unnumbered| option omits section numbers.
% It's effect is similar to using \mintinline{latex}|\section*| instead of \mintinline{latex}|\section|.
%
% Declare conditional.
%    \begin{macrocode}
\newif\ifsyllabus@unnumbered\relax
%    \end{macrocode}
% Set conditional.
%    \begin{macrocode}
\DeclareOption{unnumbered}{%
  \syllabus@unnumberedtrue%
}%
%    \end{macrocode}
%
% Process options.
% \mintinline{latex}|\relax| avoids unnecessary lookahead due to the starred form of \mintinline{latex}|\ProcessOptions|.
%    \begin{macrocode}
\ProcessOptions\relax%
%    \end{macrocode}
%
% \subsection{Packages}
% Load packages required by this one.
%
% \textsf{geometry} modifies the default margin size.
% Use a 1-inch margin on all pages.
%    \begin{macrocode}
\RequirePackage[
    margin=1in,
]{geometry}
%    \end{macrocode}
%
% \textsf{titlesec} simplifies customizing the format of section titles.
%    \begin{macrocode}
\RequirePackage{titlesec}
%    \end{macrocode}
%
% \subsection{Macros}
% This section describes the macros in the \textsf{syllabus} package.
% Most macros are redefinitions of existing macros to customize their style.
%
% \begin{macro}{logo}
%   Define the |logo| command to allow the inclusion of the university logo above the title.
%    \begin{macrocode}
\newcommand*{\logo}[1]{%
  \global\def\@logo{#1}%
}
%    \end{macrocode}
%   Declare an ``empty'' logo in case it isn't provided.
%    \begin{macrocode}
\logo{}  % empty logo
%    \end{macrocode}
% \end{macro}
%
% \begin{macro}{maketitle}
%   Redefine \mintinline{latex}|\maketitle| to include the university logo followed by the title (and optional thanks).
%    \begin{macrocode}
\renewcommand{\maketitle}{%
%    \end{macrocode}
%   Center the logo, title, author, and date.
%    \begin{macrocode}
  \begin{center}%
    \let\footnote=\thanks\relax%
    \@logo\par%
    \vspace{2.5ex}%
    \textbf{\@title}\par%
    \@author\par%
    \@date\par%
  \end{center}%
  \@thanks%
%    \end{macrocode}
%   Undefine macros related to \mintinline{latex}{\maketitle}.
%    \begin{macrocode}
  \let\logo=\relax\relax%
  \let\title=\relax\relax%
  \let\author=\relax\relax%
  \let\date=\relax\relax%
  \let\thanks=\relax\relax%
%    \end{macrocode}
%    \begin{macrocode}
}
%    \end{macrocode}
% \end{macro}
%
% \begin{macro}{section}
%   Modify \mintinline{latex}|\section| to use a bold font.
%    \begin{macrocode}
\titleformat%
    {\section}% command
    [hang]% shape
    {\bfseries}% format
    {\ifsyllabus@unnumbered\else\thesection\fi}% label
    {\ifsyllabus@unnumbered0em\else1em\fi}% sep
    {}% before-code
%    \end{macrocode}
%  Use the equivalent of \mintinline{latex}|\bigbreak| prior to the section title.
%    \begin{macrocode}
\titlespacing*%
    {\section}% command
    {\z@}% left
    {\bigskipamount}% before-sep
    {\z@}% aftersep
%    \end{macrocode}
% \end{macro}
%
% \begin{macro}{subsection}
%   Modify \mintinline{latex}|\subsection| to use a bold font.
%    \begin{macrocode}
\titleformat%
  {\subsection}% command
  [hang]% shape
  {\bfseries}% format
  {\ifsyllabus@unnumbered\else\thesubsection\fi}% label
  {\ifsyllabus@unnumbered0em\else1em\fi}% sep
  {}% before-code
%    \end{macrocode}
%  Use the equivalent of \mintinline{latex}|\medbreak| prior to the section title.
%    \begin{macrocode}
\titlespacing*%
    {\subsection}% command
    {\z@}% left
    {\medskipamount}% before-sep
    {\z@}% aftersep
%    \end{macrocode}
% \end{macro}
%
% \begin{macro}{subsubsection}
%   Modify \mintinline{latex}|\subsubsection| to use a bold font.
%    \begin{macrocode}
\titleformat%
  {\subsubsection}% command
  [hang]% shape
  {\bfseries}% format
  {\ifsyllabus@unnumbered\else\thesubsubsection\fi}% label
  {\ifsyllabus@unnumbered0em\else1em\fi}% sep
  {}% before-code
%    \end{macrocode}
%  Use the equivalent of \mintinline{latex}|\smallbreak| prior to the section title.
%    \begin{macrocode}
\titlespacing*%
    {\subsubsection}% command
    {\z@}% left
    {\smallskipamount}% before-sep
    {\z@}% aftersep
%    \end{macrocode}
% \end{macro}
%
% \iffalse
%</package>
% \fi
%
% \Finale
\endinput
