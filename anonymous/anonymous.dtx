% \iffalse meta-comment
%
% Copyright (C) 2021 by Joel Coffman
%
% This file may be distributed and/or modified under the
% conditions of the LaTeX Project Public License, either version 1.2
% of this license or (at your option) any later version.
% The latest version of this license is in:
%
%   http://www.latex-project.org/lppl.txt
%
% and version 1.2 or later is part of all distributions of LaTeX
% version 1999/12/01 or later.
%
% \fi
%
% \iffalse
%<*driver>
\documentclass{ltxdoc}

\usepackage{hyperref}
\usepackage{microtype}
\usepackage{minted}

\usepackage{email}
\usepackage{minted-doc}

\usepackage{anonymous}


% minted
\setminted{
  autogobble,
  breaklines,
}


% package version information
\input{.version}


\EnableCrossrefs
\CodelineIndex
\RecordChanges
\begin{document}
  \DocInput{anonymous.dtx}
\end{document}
%</driver>
% \fi
%
% \CheckSum{0}
%
% \CharacterTable
% {Upper-case   \A\B\C\D\E\F\G\H\I\J\K\L\M\N\O\P\Q\R\S\T\U\V\W\X\Y\Z
% Lower-case    \a\b\c\d\e\f\g\h\i\j\k\l\m\n\o\p\q\r\s\t\u\v\w\x\y\z
% Digits        \0\1\2\3\4\5\6\7\8\9
% Exclamation   \!     Double quote  \"     Hash (number) \#
% Dollar        \$     Percent       \%     Ampersand     \&
% Acute accent  \'     Left paren    \(     Right paren   \)
% Asterisk      \*     Plus          \+     Comma         \,
% Minus         \-     Point         \.     Solidus       \/
% Colon         \:     Semicolon     \;     Less than     \<
% Equals        \=     Greater than  \>     Question mark \?
% Commercial at \@     Left bracket  \[     Backslash     \\
% Right bracket \]     Circumflex    \^     Underscore    \_
% Grave accent  \`     Left brace    \{     Vertical bar  \|
% Right brace   \}     Tilde         \~}
%
%
% \changes{0.1.0}{2021/08/18}{Initial version}
%
% \GetFileInfo{anonymous.sty}
%
% \DoNotIndex{\#,\$,\%,\&,\@,\\,\{,\},\^,\_,\~,\ }
% \DoNotIndex{\@ne}
% \DoNotIndex{\advance,\begingroup,\catcode,\closein}
% \DoNotIndex{\closeout,\day,\def,\edef,\else,\empty,\endgroup}
% \DoNotIndex{\global,\let,\relax}
%
% \title{The \textsf{anonymous} package\thanks{This
% document corresponds to \textsf{anonymous~\fileversion-\version, dated \filedate.}}}
% \author{Joel Coffman\\\email{joel.coffman@usafa.edu}}
%
% \maketitle
%
% \begin{abstract}
% This package facilitates anonymizing documents (e.g., conference papers or journal articles) for double-blind review.
% In particular, it allows the anonymized and non-anonymized versions of the text to be written in parallel, and changing a package option is all that is required to use the anonymized (or non-anonymized) version.
% \end{abstract}
%
% \section{Usage}\label{section:usage}
% Load this package in the document preamble---e.g.,
% \begin{VerbatimOut}[gobble=2]{examples/usepackage.tex}
% \usepackage[anonymize]{anonymous}
% \end{VerbatimOut}
% \inputminted{latex}{examples/usepackage.tex}
% The |anonymize| option must be specified to anonymize text in the document.
%
% \DescribeMacro{\anonymize}
% The |anonymize| macro facilitate the anonymization of short portions of text.
% |\anonymize|\oarg{anonymized text}\marg{normal text} displays its optional argument when the document should be anonymized (using the |anonymize| package option) and its mandatory argument otherwise.
% For example,  ^^A Marshall College was one of the schools where Indiana Jones was a professor (https://tinyurl.com/7mscwuk4)
% \begin{VerbatimOut}[gobble=2]{examples/anonymize.tex}
% With support from \anonymize[a coeducational college]{Marshall College}\dots
% \end{VerbatimOut}
% \inputminted{latex}{examples/anonymize.tex}
% appears as
% \begin{quote}
%   \makeatletter\anonymizetrue\makeatother
%   \input{examples/anonymize.tex}
% \end{quote}
% when the |anonymize| package option has been specified and
% appears as
% \begin{quote}
%   \input{examples/anonymize.tex}
% \end{quote}
% otherwise.
% The optional argument may be omitted, in which case placeholder text is substituted.
% For example,
% \begin{VerbatimOut}[gobble=2]{examples/anonymize.tex}
% With support from \anonymize{Marshall College}\dots
% \end{VerbatimOut}
% \inputminted{latex}{examples/anonymize.tex}
% appears as
% \begin{quote}
%   \makeatletter\anonymizetrue\makeatother
%   \input{examples/anonymize.tex}
% \end{quote}
% (when the |anonymize| package option has been specified).
% While the placeholder text is not ideal for readability, it clearly identifies that the text has been changed and the reason why.
%
% \StopEventually{
%   \PrintChanges
%   \PrintIndex
% }
%
% \appendix
%
% \iffalse
%<*package>
% \fi
%
% \section{Implementation}
% This section documents the implementation of the package.
%
% Require \LaTeXe.
%    \begin{macrocode}
\NeedsTeXFormat{LaTeX2e}
%    \end{macrocode}
% Identify package and version.
%    \begin{macrocode}
\ProvidesPackage{anonymous}[%
    2023/07/17 %
    v0.2.0 %
    Anonymize documents for double-blind review%
]
%    \end{macrocode}
%
% \subsection{Options}
% Declare conditional for anonymizing content.
%    \begin{macrocode}
\newif\ifanonymize
%    \end{macrocode}
% \changes{0.1.1}{2022/04/25}{
%   Rename conditional to allow use outside package
% }
%
% Handle the \texttt{anonymize} option.
%    \begin{macrocode}
\DeclareOption{anonymize}{
  \anonymizetrue
}
%    \end{macrocode}
%
% \paragraph{Compatibility}
% Special handling for classes that also support anonymizing documents.
%
% Accept the \texttt{anonymous} option used by \textsf{acmart} as an alias for \texttt{anonymize}.
%    \begin{macrocode}
\@ifclassloaded{acmart}{
  \DeclareOption{anonymous}{
    \anonymizetrue
  }
}{}
%    \end{macrocode}
% \changes{0.2.0}{2023/07/17}{
%   Alias \texttt{anonymize} option if using \textsf{acmart} class
% }
%
% Process the options.
%    \begin{macrocode}
\ProcessOptions\relax
%    \end{macrocode}
%
% \subsection{Packages}
% Load packages required by this one.
%
% Use the \textsf{textcomp} package for angle brackets (in the |\meta| command).
%    \begin{macrocode}
\RequirePackage{textcomp}
%    \end{macrocode}
%
% \subsection{Configuration}
% Allow ``anonymized'' to be hyphenated (automatically).
%    \begin{macrocode}
\hyphenation{anon-y-mized}
%    \end{macrocode}
%
% \begin{macro}{\author}
% Omit the author(s) when the document should be anonymous.
%    \begin{macrocode}
\ifanonymize%
  \@ifclassloaded{acmart}{}{
    \def\author#1{\global\def\@author{}}%
  }
\fi
%    \end{macrocode}
% \changes{0.2.0}{2023/07/17}{
%   Do not anonymize author(s) if using \textsf{acmart} class
% }
% \end{macro}
%
% \subsection{Macros}
% This section describes the macros in the \textsf{anonymous} package.
%
% \begin{macro}{\anonymize}
% Define the |\anonymize| macro.
%    \begin{macrocode}
\newcommand*{\anonymize}[2][\anonymize@meta{anonymized for peer review}]{%
  \ignorespaces%
  \ifanonymize#1\else#2\fi%
}
%    \end{macrocode}
% \end{macro}
%
% \begin{macro}{\anonymize@meta}
% Define the |\anonymize@meta| macro.
%    \begin{macrocode}
\providecommand*{\anonymize@meta}[1]{\textlangle#1\textrangle}
%    \end{macrocode}
% \end{macro}
%
% \begin{macro}{\meta}
% If |\meta| hasn't already been defined, define it so it can be used for replacement text.
%    \begin{macrocode}
\ifdefined\meta%
\else%
  \let\meta=\anonymize@meta\relax%
\fi%
%    \end{macrocode}
% \end{macro}
%
% \iffalse
%</package>
% \fi
%
% \Finale
\endinput
